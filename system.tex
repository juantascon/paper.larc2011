%%-*-latex-*-

\section{System Overview}

Cantagallo and Yariguí electrical generation plants are own by Enecsa S.A.
They supply electric power to Ecopetrol S.A perforation facilities in the
Cantagallo area in the Magdalena Medio region. Cantagallo generation
center is geographically located in a town that shares the same name, and this
in turn is located in the Colombian department of Bolivar. Finally, the Yariguí
generation center is located 4km direction 56 grades azimuth regarding to
the Cantagallo generation center across the Magdalena river located at the
Puerto Wilches town which belongs to Colombian department of Santander.

The project involves the automation of the Cantagallo and Yariguí
generation units. It will allow to perform every supervision,
monitoring and electrical parameters control functions for both
electrical plants from the SCADA station located at the Yariguí
center. Subsequently, a network architecture is defined to connect the
protection relays, electrical parameters meters, power meters, gas
flow transmitter and the generators control panels.

The generators are QSV-91G made by Cummins~\cite{cummins:2005}, they operate 
with natural gas with a total capacity of 1.75MW for a nominal tension 
of 13.2KV. The motor in charge of propel the generator spins at 1500RPM and 
generator configuration is 4 pole alternator through gearbox for 60Hz
system frequency. A Cummins GCP integrated control has been installed 
in order to synchronize the generators, the excitation controls, governor 
and AVR. The electronic units of this system are a PCS (AVR control), a CM700
board to control the speed and a Premium PLC made by Schneider Electric
to manage the digital and analog inputs and outputs and also the
communication functions and the rest of the generator control tasks.

On the one hand the Cantagallo electrical center has two generation
units and the output tension is reduced to 4.16KV. On the other hand
the Yarigui electrical center has four generation units and offers
an output tension increased to 34.5KV. The protection relays used on
each generator and its associated transformer power are Beckwith
M-3425A. Each bay has an Schneider Electric ION6200 meter
and the output line has a protection relay SEPAM20 also by Schneider
Electric. A Landys \& Gyr meter at each generation center's output
line registers the electric power consumption. Finally, it is
essential to supervise the gas input parameters (pressure and flow)
for the generators to work properly, this is executed by a
ThermoScientific – Autopilot Pro meter.

Because the equipment operation and control of the generation centers
is done manually it is necessary to automate such processes from a
centralized unit. The system will integrate the monitoring and automatic
remote control of the equipment and the installed relays of both
generation plants to a supervision system built at the Yariguí control
room. The Substation Automation System SAS will integrate in a communication 
network the following equipment: protection, electric metering, generators' 
control units, auxiliary supervision services, outdoor transformers and each
substation's gas meter. The automation must be conceived as a modular
system providing robustness and ease of expansion and as a open system
due to the use of the plant equipment interfaces, ports and
protocols~\cite{neumann:2007}.

The Cantagallo generation center will be linked to the main control system
through broad band channel. In this way an information network will create 
a communication path to share data with the Yariguí generation center.

The solution includes the re-synchronization maneuver by using dry
contacts cables from the PLCs digital output. This also applies to
the fire alarm supervision, the opening of the generation center
access doors, trip of the gas valves by high or low pressure,
intruders alert alarms and lights off and on switching.
