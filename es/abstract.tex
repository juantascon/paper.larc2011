%%-*-latex-*-

\begin{abstract}
En el presente artículo se muestra la concepción y desarrollo de
un sistema distribuido de automatización los centros de generación
eléctricos de Cantagallo y Yarigui. Se propone una arquitectura
basada en MODBUS en la capa de dispositivos I/O y una red Ethernet
como núcleo de las comunicaciones; es importante destacar que la Central
de Cantagallo se opera de manera remota. La integración del sistema se
implementa con un SCADA basado en Vijeo Citect, en el cual predomina el
uso del protocolo Modbus en sus distintas versiones presentes en los
equipos de los centros de generación. Se presenta el software que
involucra la programación del PLC Quantum, la aplicación de la HMI y el
SCADA que será el principal componente de la automatización.
Actualmente la aplicación está funcionando satisfactoriamente en las
dos centrales de generación de ECOPETROL.
\end{abstract}
