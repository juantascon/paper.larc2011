%%-*-latex-*-

\section{Introduction}

La necesidad de monitorear remotamente las subestaciones eléctricas
entre otras por su ubicación geográfica destaca como de vital
importancia la automatización y control centralizado modernizando de
esta forma la instalación, aumentando la eficiencia y garantizando
la disponibilidad del servicio.

El protocolo Ethernet se ha proliferado altamente en los IEDs 
(Intelligent Electronic Devices) presentes en los sistemas de 
subestaciones automatizadas (SAS) [3], donde se ha establecido 
como la tecnología lider por su confiabilidad e integrabilidad.
En aplicaciones modernas de automatización ha alcanzado el 
punto de ser usado para ejecución de aplicaciones en tiempo real [2].

La tendencia es la misma en equipos de control como PLC, transductores
inteligentes y equipos de comunicacion como gateway que en la mayoria de 
casos siempre esta disponible un puerto de comunicacion con ethernet con
la opcion de utilizar tcp/ip o udp [4]. Por tal razón no puede pasarse por
alto el uso de esta tecnología como base de red y nucleo principal de un
sistema automatizado.

El objetivo del presente artículo es mostrar las labores de ingeniería,
suministro, montaje, pruebas y puesta en servicio para llevar a cabo la 
automatización de las plantas de generación Cantagallo y Yarigui propiedad de 
ECOPETROL, para lo cual se pretende integrar los equipos y relés existentes 
en las plantas de generación a un sistema de supervisión implementado en la 
sala de control de Yarigui, con el propósito de obtener un funcionamiento 
adecuado del monitoreo, el control automático, remoto de los equipos y el 
sistema general de Cantagallo.

Una de las mayores dificultades para esta aplicación es el control y monitoreo 
remoto de la planta de generación de Cantagallo,  en el cual se debe establecer 
un canal confiable y seguro que garantice la comunicación y disponibilidad del 
sistema en todo momento, para esto se selecciono un canal bajo el estándar 802.11 
con radios de alcance de hasta 20km que es integrado al nucleo de la red Ethernet.

El proyecto en su totalidad se encuentra culminado y en operación destacando el 
trabajo de ingeniería desarrollado netamente por ingenieros Colombianos. Esperando 
que sea tomado como referencia para realizar distintas labores de automatización o 
en el caso presentarse la necesidad de una adecuada selección de arquitectura de control.

Este artículo se divide en cuatro partes. En la primera se presenta una breve descripción 
del sistema y la tarea de automatización que se desea solucionar. La segunda muestra la 
arquitectura y el diseño de ingeniería planteado.  La tercera ofrece un esquema general 
de la aplicación software desarrollada y del SCADA programado para las tareas de 
automatización. Finalmente en la cuarta parte se presenta el resultado de las pruebas 
y de la puesta en servicio.
