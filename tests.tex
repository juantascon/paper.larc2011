%%-*-latex-*-

\section{Running Tests}

The system running tests implies the verification of every supervision
data and the controls presents at the HMI and the developed SCADA.

As a supervision test schema a checklist was made including every
signal from every user interface. Every SCADA data signal returned
successful presenting an average update time of 1 second per data.

One important point is the communication link stability at Cantagallo
which proved the high efficiency during the supervision and control
tests.

An special display was created at the SCADA in order to monitor the
several communication networks available in the system. This shows both
a communication architecture at the generation center and the link state
for each device. Finally it helps to the maintenance and failure report
tasks.

Control tests executed and ordered for each generation center included
the tests for the HMI and the Cantagallo substation SCADA. The general
tests schema considered the following items:

\begin{itemize}
\item Control tests for each bay, including start, bar synchronization and stop.
\item Control tests for the output line interrupter: opening and closing.
\item Protection relays reset and interrupter shot sticking.
\item Re-synchronization with public network (generation center operation at
  network mode) transition to operation in Island mode and start of
  machines in dead bar.
\end{itemize}

The control signals were proved for each case altogether, resulting in effective
running and visualization tests.

Finally, at full operation, the Cantagallo-Yariguí communication channel
was supervised for a period of 24 hours. The results indicate that the
channel has an average packet loss rate of 1.0\%. Such results assure
the communications' stability.
