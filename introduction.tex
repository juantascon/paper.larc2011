%%-*-latex-*-

\section{Introduction}

%TODO: poner las referencias

Because of their geographical location it is necessary to remotely
monitor the electrical substations. It is of vital importance the
automation and centralized control, thus modernizing the facilities.
The results would be a higher efficiency and a higher service
availability.

%~\cite{Kahney:1983}

Ethernet protocol has been incredibly spread into IEDs (Intelligent 
Electronic Devices) at the Substation Automation Systems (SAS).
In such scenarios, Ethernet is the leading technology for its reliability
and integrability~\cite{pozzuoli:2003}. On modern automation 
implementations it has reach the point to be used for execution of 
real-time applications~\cite{bello:2001}. Control hardware including 
PLC, smart transducers and communication equipment such as gateways 
include, most of the time, an Ethernet communication port with an option 
to use either TCP/IP or UDP~\cite{viegas:2006}. For these reasons
this technology can not be overlooked as network base and main core
for an automation system.

The objective of this article is to show the engineering labors to
carry out the automation of ECOPETROL's Cantagallo and Yariguí
generation plants. Such labors include supply, assembly, tests and
come into service. To do this, the existent equipment and relays must
be integrated to a supervision system implemented at the Yariguí
control room. It is expected to obtain a much more accurate working of
the monitoring, automatic and remote equipment control and in general
of the entire Cantagallo system.

One of the biggest obstacles for this implementation is to be able to
remotely control and monitor the Cantagallo generation plant. The
hardest part is to establish a reliable and secure channel to
guarantee, at every moment, the communication and availability of the
system. To solve this, an IEEE 802.11 conformant channel was installed
with a range of 20KM integrated to the Ethernet network core.

The project has been completed and it is currently in operation. It is
highlighted the engineering work done entirely by Colombian engineers.
It is expected to be considered as a reference for future automation
labors and as a future option for control architectures.

This paper is divided in four parts. The first section introduces a
quick description of the system and the automation problem. The second
shows the proposed architecture and design. The third part offers a
general schema of the programmed software and SCADA. Finally, the last
part presents the tests and come in service results.
